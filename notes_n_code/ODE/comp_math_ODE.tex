\documentclass[a4paper,11pt]{article} % добавить leqno в [] для нумерации слева

%%% Работа с русским языком
\usepackage{cmap}					% поиск в PDF
\usepackage{mathtext} 				% русские буквы в формулах
\usepackage[T2A]{fontenc}			% кодировка
\usepackage[utf8]{inputenc}			% кодировка исходного текста
\usepackage[russian,english]{babel}	% локализация и переносы

%%% Дополнительная работа с математикой
\usepackage{amsmath,amsfonts,amssymb,amsthm,mathtools} % AMS
\usepackage{icomma} % "Умная" запятая: $0,2$ --- число, $0, 2$ --- перечисление

%% Номера формул
\mathtoolsset{showonlyrefs=true} % Показывать номера только у тех формул, на которые есть \eqref{} в тексте.

%% Шрифты
\usepackage{euscript}	 % Шрифт Евклид
\usepackage{mathrsfs} % Красивый матшрифт

%% Свои команды
\DeclareMathOperator{\sgn}{\mathop{sgn}}
\newcommand{\eqdef}{\stackrel{\rm def}{=}} 


%% Перенос знаков в формулах (по Львовскому)
\newcommand*{\hm}[1]{#1\nobreak\discretionary{}
	{\hbox{$\mathsurround=0pt #1$}}{}}

%%% Заголовок
\author{Alexander Petrov}
\title{Computational Mathematics: Ordinary Differential Equations}


%%% Теоремы
\theoremstyle{plain} % Это стиль по умолчанию, его можно не переопределять.
\newtheorem{theorem}{Th}[section]
\newtheorem{lemma}{Le}[section]
\newtheorem{proposition}{Утверждение}[section]
\newtheorem{note}{Note}[section]


\theoremstyle{definition} % "Определение"
\newtheorem{cor}{Следствие}[theorem]
\newtheorem{problem}{Задача}[section]
\newtheorem{defin}{Def}[section]
\newtheorem{pr}{Pr}[defin]
\newtheorem{ex}{Ex}[section]
\newtheorem{n}{№}

\theoremstyle{remark} % "Примечание"
\newtheorem*{nonum}{Решение}
\newtheorem*{oboz}{Обозначение}
\newtheorem*{ev}{Доказательство}
\newtheorem*{id}{Идея доказательства}

\usepackage{mathrsfs}

%%% Работа с картинками
\usepackage{graphicx}  % Для вставки рисунков
\graphicspath{{Mimage/}}  % папки с картинками
\setlength\fboxsep{3pt} % Отступ рамки \fbox{} от рисунка
\setlength\fboxrule{1pt} % Толщина линий рамки \fbox{}
\usepackage{wrapfig} % Обтекание рисунков и таблиц текстом

\usepackage{arcs}

\usepackage{geometry} % Простой способ задавать поля
\geometry{top=25mm}
\geometry{bottom=25mm}
\geometry{left=20mm}
\geometry{right=20mm}

\usepackage{fancyhdr} % Колонтитулы
\pagestyle{fancy}
\renewcommand{\headrulewidth}{0.3mm}  % Толщина линейки, отчеркивающей верхний колонтитул
\lfoot{}
\rfoot{}
\rhead{}
\chead{}
\lhead{}

\begin{document}

\maketitle

\newpage

\tableofcontents

\newpage

\section{Main definitions.}

Let us consider a problem with given initial conditions (Cauchy's problem):
\begin{equation}
\label{cauchy_problem}
	\left\{
	\begin{aligned}
	& \mathscr{D} u = f(x, u), \\
	& u(x_0) = u_0, \\
	& x \in [0, X],
	\end{aligned}
	\right.
\end{equation}
where $\mathscr{D}$ is some differential operator (generally not linear). Our purpose is to discretize domain and equation: 
\begin{equation}
\label{cauchy_problem_discretized}
\left\{
\begin{aligned}
& D y = f_h, \\
& y_0 = u^0, \\
& \{x_i\}_{i=0}^N:\ h_i = x_{i+1} - x_i
\end{aligned}
\right.
\end{equation}

Let us denote $[u]_h$ as values of the exact solution $u(x)$ at grid nodes $\{x_i\}_{i=0}^N$, and $h = \underset{i}{\max}\ h_i$.

\begin{defin}
	Solution of \eqref{cauchy_problem_discretized} \textbf{converges} to solution of \eqref{cauchy_problem} if 
	\begin{equation}
		\Vert y - [u]_h \Vert_Y \rightarrow 0,\ h \rightarrow 0
	\end{equation}
\end{defin}

\begin{note}
	It is almost imposible and useless to prove convergense directly. Indeed, if you have $[u]_h$ why do you need to solve \eqref{cauchy_problem_discretized}?
\end{note}

\begin{defin}
	\textbf{Residual} of \eqref{cauchy_problem_discretized} is:
	\begin{equation}
		r_h = D [u]_h - f_h
	\end{equation}
\end{defin}

\begin{defin}
	Solution of \eqref{cauchy_problem_discretized} \textbf{approximates} solution of \eqref{cauchy_problem} if 
	\begin{equation}
		\Vert r_h \Vert_{F_h} \rightarrow 0,\ h \rightarrow 0.
	\end{equation}
	 If $\exists\ C_1 > 0:\ \Vert r_h \Vert_{F_h} \le C_1 h^k$, the number k is called the order of approximation.
\end{defin}

\begin{defin}
	Solution of \eqref{cauchy_problem_discretized} is \textbf{stable} if $\exists\ \varepsilon,\ h_0:\ \forall h \le h_0,\ \forall \varepsilon^{(1)}, \varepsilon^{(2)}: \Vert \varepsilon^{(i)} \Vert \le \varepsilon,\ i = 1,2$ solutions of 
	\begin{equation}
		\begin{aligned}
		& D y^{(1)} = f_h + \varepsilon^{(1)} \\
		& D y^{(2)} = f_h + \varepsilon^{(2)}
		\end{aligned}
	\end{equation}
	<<do not differ much>>:
	\begin{equation}
		\Vert y^{(2)} - y^{(1)} \Vert \le C_2 (\Vert \varepsilon^{(1)}\Vert + \Vert \varepsilon^{(2)}\Vert)
	\end{equation}
\end{defin}

\section{Runge-Kutta methods}

\begin{defin}
	Explicit Runge-Ketta method with $s$ stages and defining parameters
\end{defin}


\end{document}